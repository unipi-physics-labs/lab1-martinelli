\begin{exercises}

\exercise{Verificare {\itshape sperimentalmente} la (\ref{eq:ProbVenn})
nel caso del lancio di una moneta (verificare, cio\`e, che il rapporto tra
il numero di teste $n$ ed il numero di lanci $N$ tende ad $\frac{1}{2}$
al crescere di $N$).

\hint{operativamente si pu\`o costruire un grafico cartesiano in
cui, dopo ogni lancio, si pone $N$ sull'asse delle $x$ e $n/N$ su quello
delle $y$.}}

\exercise{Sia data una scatola contenente $6$ palline rosse, $3$ blu e $5$
bianche. Supponiamo di estrarre una pallina a caso dalla scatola; calcolare
la probabilit\`a che essa sia blu o bianca.}
\answer{\frac{4}{7}}

\exercise{Nel lancio di una moneta qual \`e la probabilit\`a di avere
una successione di cinque teste, seguita da una croce?}
\answer{\frac{1}{64}}

\exercise{Nel lancio di una moneta, qual \`e la probabilit\`a che la
\emph{prima} croce esca esattamente al $k$-esimo tentativo (cio\`e che
escano $k-1$ teste in successione, seguite da una testa)?}
\answer{\frac{1}{2^k}}

\exercise{\label{exr:TreNelLancioDado}Consideriamo il lancio di una dado a
sei facce. Qual \`e la probabilit\`a che un numero fissato (ad esempio il $3$)
esca per la prima volta al $k$-esimo tentativo (cio\`e che le prime $k-1$
uscite siano diverse da $3$ e la $k$-esima sia $3$)?}
\answer{\frac{1}{6} \cdot \left( \frac{5}{6} \right)^{k-1}}

\exercise{Con riferimento all'esempio \ref{esem:Autobus}, quanti minuti
dovremo aspettare l'autobus alla fermata, in media?

\hint{si tratta semplicemente di calcolare la media della distribuzione
scritta esplicitamente nell'esempio usando la definizione
(\ref{eq:Media}).}}
\answer{5}

\exercise{Provate a rappresentare le funzioni di distribuzione per la somma 
delle uscite di uno, due e tre dadi e confrontate i grafici.}

\exercise{\label{eser:Lavagna}Un gruppo di studenti (ognuno dei quali
\`e identificato da un numero progressivo $i$) misura ad occhio ed
indipendentemente la lunghezza della lavagna dell'aula A. Vengono registrati
i seguenti risultati (in metri)%
\footnote{
Dati realmente ottenuti il 9 Gennaio 1990.
}%
:

\listtable{10}{Lunghezza}{cm}
{$6$    & $4.8$  & $6.5$  & $5.62$ & $3$    &
 $7.5$  & $5.5$  & $4$    & $5.5$  & $5$    \\
 $4.5$  & $5$    & $4.75$ & $4$    & $5.8$  &
 $5$    & $5$    & $5.5$  & $4.6$  & $4.7$  \\
 $4.5$  & $5.3$  & $4.75$ & $2.80$ & $4.50$ &
 $5$    & $6.1$  & $4$    & $5.5$  & $4.6$  \\
 $6.60$ & $5.80$ & $5.4$  & $5.5$  & $7.5$  &
 $5.9$  & $10$   & $5.8$  & $3.7$  & $5.4$  \\
 $5.5$  & $5$    & $5.5$  & $5.8$  & $5$    &
 $5.5$  & $5.7$  & $5$    & $6.2$  & $6.1$  \\
 $3.45$ & $7.5$  & $5.40$ & $3.5$  & $6.2$  &
 $5.18$ & $4.5$  & $4.80$ & $5.30$ & $5.5$  \\
 $6$    & $5.3$  & $6$    & $5$    & $6.20$ &
 $6$    & $5$    & $5$    & $6.5$  & $5.5$  \\
 $5$    & $5.5$  & $5.35$ & $5.2$  & $5.8$  &
 $4.5$  & $5.45$ & $6.5$  & $6$    & $5$    \\
 $6$    & $5.75$ & $5.25$ & $5.6$  & $5.6$  &
 $5.9$  & $5$    & $6.5$  & $8$    & $5.40$ \\
 $4$    & $3.2$  & $5.60$ & $6.05$ & $5.5$ & $6.5$\\
}
 
\noindent Rappresentare questi dati mediante un istogramma.}

\exercise{Dimostrare che:
$$
\dsum{c}{k}{1}{n} = nc
$$}

\exercise{Dimostrare che:
$$
\dsum{k}{k}{1}{n} = \frac{n(n+1)}{2}
$$

\hint{sommare i termini a due a due (il primo con l'ultimo, il secondo
con il penultimo e cos\`i via). Che cosa succede?}}

\exercise{Dimostrare che:
$$
\dsum{k^2}{k}{1}{n} = \frac{n(n+1)(2n+1)}{6}
$$

\hint{procedere per induzione.}}

\exercise{\label{exr:SommaAAllaKappa}Si consideri un numero reale positivo
$0 < a < 1$.
Calcolare esplicitamente la quantit\`a:
$$
\dsum{a^k}{k}{1}{n} = a + a^2 + a^3 + \ldots + a^n
$$

\hint{si scriva esplicitamente la somma e si moltiplichi numeratore e
denominatore per $1 - a$. Si svolga dunque il prodotto al numeratore
(un numero cospicuo di termini dovrebbe elidersi).}}
\answer{a \cdot \frac{(1 - a^n)}{(1 - a)}}

\exercise{Si consideri un numero reale positivo $0 <a < 1$.
Utilizzando il risultato dell'esercizio precedente si calcoli la quantit\`a:
$$
\dsum{a^k}{k}{1}{\infty}
$$

\hint{si calcoli il limite, per $n$ che tende ad infinito, dell'espressione
ottenuta nell'esercizio precedente.}}
\answer{\frac{a}{(1 - a)}}

\exercise{Si consideri un numero reale positivo $0 < a < 1$.
Calcolare esplicitamente la quantit\`a:
$$
\dsum{ka^k}{k}{1}{n} = a + 2a^2 + 3a^3 + \ldots + na^n
$$

\hint{si applichi il suggerimento dell'esercizio \ref{exr:SommaAAllaKappa}.
Questa volta i termini non si elidono, ma applicando il risultato
ottenuto nell'esercizio stesso\ldots}}
\answer{a \cdot \frac{(1 - a^n)}{(1 - a)^2} - \frac{na^{n+1}}{(1 - a)}}

\exercise{\label{exr:SommaKappaAAllaKappaInf}Si consideri un numero reale
positivo $0 < a < 1$.
Utilizzando il risultato dell'esercizio precedente si calcoli la quantit\`a:
$$
\dsum{ka^k}{k}{1}{\infty}
$$}
\answer{\frac{a}{(1-a)^2}}

\exercise{Verificare che la particolare distribuzione triangolare definita
nell'esempio \ref{esem:MediaDistTriangolare} \`e normalizzata.}

\exercise{Si dimostri che il valore di aspettazione di una costante \`e uguale
al valore della costante stessa.

\hint{si scriva esplicitamente il valore di aspettazione e si sfrutti la
condizione di normalizzazione della funzione di distribuzione.}}

\exercise{Si consideri di nuovo l'esercizio \ref{exr:TreNelLancioDado}.
Quanti lanci di un dado a sei facce dobbiamo aspettare, in media, perch\'e
esca un $3$?

\hint{sappiamo gi\`a, dall'esercizio \ref{exr:TreNelLancioDado}, la
probabilit\`a che il primo $3$ esca esattamente al $k$-esimo lancio.
Si tratta allora di valutare il valore di aspettazione $\expect{k}$ per la
specifica funzione di distribuzione. Sar\`a utile anche il risultato
dell'esercizio \ref{exr:SommaKappaAAllaKappaInf}.}}
\answer{6}

\exercise{\label{eser:XQuadroUnDado}Sia la variabile casuale $x$ l'uscita
di un dado equo a sei facce. Si calcoli il valore di aspettazione
$\expect{x^2}$.}
\answer{\frac{91}{6}}

\exercise{Si verifichi la validit\`a della (\ref{eq:Varianza2})
sfruttando il risultato dell'esercizio precedente.}

\hint{si usino gli esempi \ref{esem:MediaUnDado} e \ref{esem:VarianzaUnDado}.}

\exercise{Si ripeta l'esercizio \ref{eser:XQuadroUnDado} nel caso in cui la
variabile casuale $x$ sia la somma delle uscite nel lancio di due dadi
(si calcoli il valore di aspettazione $\expect{x^2}$).}
\answer{\frac{329}{6}}

\exercise{Si verifichi la validit\`a della (\ref{eq:Varianza2})
sfruttando il risultato dell'esercizio precedente.

\hint{si sfruttino in questo caso gli esempi \ref{esem:MediaDueDadi} e
\ref{esem:VarianzaDueDadi}}.}

\exercise{\label{exr:Mom3}Dimostrare che per una distribuzione arbitraria il
momento di ordine $3$ centrato nel valor medio vale:
$$
\mommean{3} = \expect{x^3} - 3\mu\sigma^2 - \mu^3
$$

\hint{scrivere esplicitamente la definizione di $\mommean{3}$ e sfruttare
la linearit\`a dell'operatore valore di aspettazione, insieme alla
(\ref{eq:Varianza2}).}}

\exercise{Dimostrare che per una funzione di distribuzione simmetrica
rispetto al valor medio il coefficiente di asimmetria (\ref{eq:Skewness})
\`e nullo.}

\end{exercises}
