\thispagestyle{empty}

\chapter*{Introduzione}
\pdfbookmark[0]{Introduzione}{introduzione_seconda}

Non vi \`e dubbio che \emph{condensare} in poche decine di
pagine le informazioni necessarie ad un uso (anche solo vagamente)
proficuo del calcolatore per l'analisi di dati scientifici sia
una pretesa non da poco.
Da una parte, infatti, la quantit\`a di informazioni disponibile \`e
virtualmente sconfinata, il che rende la selezione del materiale, di per
s\'e, difficoltosa. Anche una volta scelto il materiale, si pone poi il
problema di presentarlo in maniera comprensibile, organica e, per quanto
possibile, originale.

L'idea di base attorno alla quale questa seconda parte \`e organizzata \`e
sostanzialmente la definizione delle nozioni e delle competenze che lo
studente deve acquisire perch\'e il primo incontro con il calcolatore si
riveli fruttuoso. Nozioni che si riassumono
essenzialmente nella capacit\`a utilizzare il \emph{computer} per realizzare
grafici, eseguire \emph{fit} numerici e presentare degnamente i risultati
dell'analisi. I pacchetti \emph{software} che abbiamo selezionato a questo
scopo sono \gnuplot\ e \LaTeX, supportati nella nostra scelta dalla loro ampia
diffusione all'interno della comunit\`a scientifica e dalla semplicit\`a
di utilizzo (oltre alla nostra predilezione per il \emph{free software}).

Abbiamo anche ritenuto opportuno preporre alla descrizione di questi pacchetti
alcune nozione di base sull'uso del calcolatore; una sorta di manuale di
sopravvivenza per utilizzatori (molto) inesperti del sistema operativo Linux.

I riferimenti bibliografici dovrebbero essere pi\`u che sufficienti per colmare
le lacune delle pagine che seguono e soddisfare la curiosit\`a dei pi\`u
esigenti. 

\clearpage
\thispagestyle{plain}
