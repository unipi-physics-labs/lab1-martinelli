\begin{exercises}

\exercise{Seguendo le sezioni \ref{subsec:FitMinQuadCostante} e
\ref{subsec:FitMinQuadLineare} ricavare con il metodo dei minimi quadrati
il miglior valore di $a$ per la funzione:
$$
f(x, a)=a x
$$}

\exercise{Seguendo le sezioni \ref{subsec:FitMinQuadCostante} e
\ref{subsec:FitMinQuadLineare} ricavare con il metodo dei minimi quadrati
il miglior valore di $a$ per la funzione:
$$
f(x, a)=a x^2
$$}

\exercise{Seguendo le sezioni \ref{subsec:FitMinQuadCostante} e
\ref{subsec:FitMinQuadLineare} ricavare con il metodo dei minimi quadrati
il miglior valore di $a$ per la funzione:
$$
f(x)=a e^{-x}
$$}

\exercise{Calcolare esplicitamente la media pesata (con relativa incertezza)
e la media aritmetica dei valori dell'indice di rifrazione dell'acqua
riportati nell'esempio \ref{esem:MediaPesata} e verificare che i
risultati dell'esempio stesso sono corretti.}

\exercise{Con riferimento all'esempio \ref{esem:LeggeOrariaMinimiQuadrati},
eseguire un fit dei minimi quadrati utilizzando i risultati ottenuti
nella sezione \ref{subsec:FitMinQuadLineare} e verificare che i risultati
coincidano con quelli forniti alla fine dell'esercizio stesso---i dati sono
riportati nella tabella in figura \ref{fig:LeggeOrariaMinimiQuadrati}.}

\exercise{Con riferimento all'esempio \ref{esem:LeggeOrariaMinimoChiQuadro},
eseguire un fit del minimo $\chi^2$ utilizzando i risultati ottenuti
nella sezione \ref{sec:MinChiQuadroLineare} e verificare che i
risultati coincidano con quelli forniti alla fine dell'esercizio
stesso---i dati sono riportati nella tabella in figura
\ref{fig:LeggeOrariaMinChiQuadro}.}

\exercise{Discutere se per i dati riportati nell'esercizio \ref{eser:Lavagna}
\`e accettabile l'ipotesi di distribuzione Gaussiana (qual \`e il livello di
significativit\`a).}

\end{exercises}
