\thispagestyle{empty}

\chapter*{Introduzione}
\pdfbookmark[0]{Introduzione}{introduzione_prima}

\noindent Quando un fisico vuole misurare qualche cosa si pone sostanzialmente
due problemi:
\begin{numlist}
\item Come si misura, cio\`e come si pu\`o ottenere il risultato.
\item Quanto \`e giusto il risultato, o (il che \`e lo stesso)
quanto \`e sbagliato.
\end{numlist}
Si noti che il secondo punto \`e essenziale perch\'e altrimenti
anche il primo perde gran parte del suo significato.
Infatti  il risultato di una misura \`e soltanto una approssimazione
o stima del valore della grandezza specifica che si vuole misurare
(il \emph{misurando}) ed il risultato \`e completo quando \`e accompagnato da
una quantit\`a che definisce l'incertezza.
Ci sono due situazioni tipiche che limitano la precisione delle misure:
\begin{numlist}
\item Lo strumento ha dei limiti. Si parla allora di
\emph{incertezza strumentale} ed il risultato delle misure in genere \`e
costante. Per esempio nella misura di
un tavolo con un metro a nastro si ottiene sempre lo stesso risultato entro la
risoluzione dello strumento cio\`e entro il millimetro.
\item Lo strumento \`e molto raffinato, ma il fenomeno che si studia sfugge al
controllo completo (almeno in qualche sua caratteristica secondaria).
Allora si hanno risultati diversi da misura a misura: si
dice che si hanno \emph{fluttuazioni casuali}.
\end{numlist}
\`E chiaro che nella fisica moderna ha molta pi\`u importanza la seconda
situazione perch\'e quasi
sempre si lavora al limite delle tecniche strumentali.
Dovremo quindi studiare come ci si deve comportare quando i risultati
delle misure fluttuano in modo imprevedibile e casuale: gli strumenti
matematici necessari sono la probabilit\`a e la statistica.

\clearpage
\thispagestyle{plain}
