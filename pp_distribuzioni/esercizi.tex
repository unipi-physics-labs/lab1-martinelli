\begin{exercises}

\exercise{Costruire sperimentalmente la distribuzione
$\binomialpdf{k}{4, \tinyfrac{1}{6}}$ operando in questo modo:
\begin{numlist}
\item Fare $N$ prove, ogni prova sia un lancio di 4 dadi.
\item Contare quanti 5 escono in ciascuna prova.
\item Contare quante prove contengono $k$ $5$.
\item Dividere per il numero di prove, in modo da ottenere la frequenza
relativa, che \`e una stima della probabilit\`a.
\item Rappresentare le frequenze relative e confrontarle con quelle teoriche.
\end{numlist}}

\exercise{\label{exr:RicorrenzaBinomiale}Sia $k$ una variabile casuale
con funzione di distribuzione binomiale. Si dimostri che vale la seguente
relazione di ricorrenza:
$$
\expect{k^{m+1}} = p(1-p) \cdot \frac{\ud}{\ud p} \, \expect{k^m} +
np \cdot \expect{k^m}
$$

\hint{si scriva esplicitamente la derivata rispetto a $p$ di $\expect{k^m}$:
$$
\frac{\ud}{\ud p} \, \expect{k^m} =
\frac{\ud}{\ud p} \, \dsum{k^m \cdot \binomialpdf{k}{n, p}}{k}{0}{n} =
\frac{\ud}{\ud p} \, \dsum{k^m \cdot \binom{n}{k}p^k (1-p)^{n-k}}{k}{0}{n}
$$
Si porti dunque la derivata dentro il segno di sommatoria e si
esegua la derivata stessa termine per termine. A questo punto non dovrebbe
essere troppo complicato provare la tesi.}}

\exercise{\label{exr:Ek2Binomiale}Sia $k$ una variabile casuale
con funzione di distribuzione binomiale. Si utilizzi il risultato
dell'esercizio \ref{exr:RicorrenzaBinomiale} per calcolare $\expect{k^2}$,
sapendo che $\expect{k} = np$.}
\answer{np(np - p + 1)}

\exercise{\label{exr:Ek3Binomiale}Sia $k$ una variabile casuale
con funzione di distribuzione binomiale. Si utilizzino i risultati degli
esercizi \ref{exr:RicorrenzaBinomiale} e \ref{exr:Ek2Binomiale} per calcolare
$\expect{k^3}$.}
\answer{np(n^2p^2 + 2p^2 - 3np^2 + 3np - 3p + 1)}

\exercise{Calcolare il coefficiente di asimmetria $\gamma_1$ per la
distribuzione binomiale.

\hint{Si usino i risultati degli esercizi \ref{exr:Ek3Binomiale} e 
\ref{exr:Mom3}, oltre ovviamente ai fatti noti sulla distribuzione
binomiale.}}
\answer{\frac{1 - 2p}{\sqrt{np(1 - p)}}}

\exercise{Calcolare il limite per grande $n$ del coefficiente di asimmetria
$\gamma_1$ della distribuzione binomiale. Commentare.}
\answer{0}

\exercise{\label{exr:RicorrenzaPoisson}Sia $k$ una variabile casuale
con funzione di distribuzione Poissoniana. Si dimostri che vale la
seguente relazione di ricorrenza:
$$
\expect{k^{m+1}} = \mu \cdot \left( \frac{\ud}{\ud\mu} \, \expect{k^m} +
\expect{k^m}\right)
$$

\hint{si scriva esplicitamente la derivata rispetto al valor medio $\mu$ di
$\expect{k^m}$:
$$
\frac{\ud}{\ud\mu} \, \expect{k^m} =
\frac{\ud}{\ud\mu} \, \dsum{k^m \cdot \poissonpdf{k}{\mu}}{k}{0}{\infty} =
\frac{\ud}{\ud\mu} \, \dsum{k^m \cdot \frac{\mu^k}{k!}\,e^{-\mu}}{k}{0}{\infty}
$$
Si porti dunque la derivata dentro il segno di sommatoria e si
esegua la derivata stessa termine per termine. A questo punto non dovrebbe
essere troppo complicato provare la tesi.}}

\exercise{\label{exr:Ek2Poisson}Sia $k$ una variabile casuale
con funzione di distribuzione Poissoniana. Si utilizzi il risultato
dell'esercizio \ref{exr:RicorrenzaPoisson} per calcolare $\expect{k^2}$,
sapendo che $\expect{k} = \mu$.}
\answer{\mu + \mu^2}

\exercise{\label{exr:Ek3Poisson}Sia $k$ una variabile casuale
con funzione di distribuzione Poissoniana. Si utilizzino i risultati degli
esercizi \ref{exr:RicorrenzaPoisson} e \ref{exr:Ek2Poisson} per calcolare
$\expect{k^3}$.}
\answer{\mu + 3\mu^2 - \mu^3}

\exercise{Calcolare il coefficiente di asimmetria $\gamma_1$ per la
distribuzione di Poisson.

\hint{Si usino i risultati degli esercizi \ref{exr:Ek3Poisson} e 
\ref{exr:Mom3}, oltre ovviamente ai fatti noti sulla distribuzione di
Poisson.}}
\answer{\frac{1}{\sqrt{\mu}}}

\exercise{Calcolare il limite per grande $\mu$ del coefficiente di asimmetria
$\gamma_1$ della distribuzione di Poisson. Commentare.}
\answer{0}

\exercise{Calcolare la semilarghezza a met\`a altezza per una funzione
di distribuzione uniforme scritta nella forma (\ref{eq:Uniforme}).}
\answer{\frac{(b-a)}{2}}

\exercise{Calcolare la mediana $\median$ per una distribuzione esponenziale
nella forma (\ref{eq:Esponenziale}).}
\answer{\frac{\ln(2)}{\lambda}}

\exercise{Calcolare la semilarghezza a met\`a altezza per una distribuzione
esponenziale nella forma (\ref{eq:Esponenziale}).}
\answer{\frac{\ln(2)}{2\lambda}}

\exercise{Sia $x$ una variabile casuale continua con funzione di distribuzione
esponenziale, nella forma (\ref{eq:Esponenziale}). Si calcoli $\expect{x^3}$.}
\answer{\frac{6}{\lambda^3}}

\exercise{Si calcoli il valore del coefficiente di asimmetria
(\ref{eq:Skewness}) per la distribuzione
esponenziale nella forma (\ref{eq:Esponenziale}).

\hint{si usi il risultato dell'esercizio precedente, insieme a quello
dell'esercizio \ref{exr:Mom3}.}}
\answer{2}

\exercise{Si calcoli la moda della distribuzione del $\chi^2$ in funzione
del numero di gradi di libert\`a (si calcoli, cio\`e, il valore di
$x$ per cui la funzione di distribuzione $\chisquarepdf{x}{n}$ assume
il suo valore massimo).

\hint{si derivi $\chisquarepdf{x}{n}$ rispetto ad $x$ e si ponga la
derivata uguale a zero.}}
\answer{n-2}

\exercise{Si dimostri che per $n = 2$ la distribuzione del $\chi2$ 
$\chisquarepdf{x}{n}$ si riduce ad una distribuzione esponenziale.}

\exercise{Sfruttando i risultati degli esercizi precedenti ed i fatti
noti dal corpo del capitolo, si compili una tabella in cui, per
le principali distribuzioni continue note, siano riportate la
deviazione standard $\sigma$, la semilarghezza a met\`a altezza
$\hwhm$ ed il fattore di proporzionalit\`a $\varepsilon$ tra le due, dato da:
$$
\hwhm = \varepsilon \sigma
$$
Si verifichi in particolare che, per ciascuna delle distribuzioni,
$\varepsilon$ \`e dell'ordine dell'unit\`a.}

\end{exercises}
