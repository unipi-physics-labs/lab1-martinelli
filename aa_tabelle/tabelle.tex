\chapter{Tavole numeriche}
\label{chap:TavoleNumeriche}

\section{Definizione della funzione \texorpdfstring{$\erf(x)$}{erf}}
\label{app:Erf}

La funzione $\erf$ (nota anche come {\itshape error function}) \`e
solitamente definita%
\footnote
{
Purtroppo alcuni autori preferiscono definire la funzione erf come:
\begin{equation}\label{eq:Erf}
\Erf(x) = \frac{2}{\sqrt{\pi}} \dintegral{e^{-\xi^2}}{\xi}{0}{x}
\end{equation}
che \`e legata alla \ref{eq:erf} da un semplice cambio di variabile:
\begin{equation}
\erf(x) = \frac{1}{2} \Erf \left( \frac{x}{\sqrt{2}} \right)
\end{equation}
Prima di utilizzare delle tavole numeriche \`e dunque essenziale
verificarne il significato esatto.
}
come:
\begin{equation}\label{eq:erf}
\erf(x) = \frac{1}{\sqrt{2\pi}} \dintegral{e^{-\xi^2/2}}{\xi}{0}{x}
\end{equation}
ed \`e essenzialmente l'integrale, eseguito tra 0 ed un estremo
variabile $x$, di una distribuzione Gaussiana di media 0 e deviazione
standard 1:
\begin{equation}
\erf(x) = \dintegral{G(z)}{z}{0}{x}
\end{equation}
\panelfig
{\onebyonetexfig{./aa_tabelle/figure/erf.tex}}
{Grafico della funzione $\erf(x)$ definita dall'equazione \ref{eq:erf}.}
{fig:Erf}

\begin{exemplify}

\example{$\erf(1) = 0.3413$. Questo significa che la probabilit\`a che il
valore di una variabile Gaussiana standard sia compreso tra $0$ ed $1$ vale
$0.3413$.\\
Il risultato pu\`o essere facilmente generalizzato ad una qualsiasi
distribuzione Gaussiana: la probabilit\`a che la variabile sia compresa tra
$\mu$ ($0$ per la variabile standard) e $\mu + \sigma$ ($0 + 1 = 1$ per la
variabile standard) vale $0.3413$.}

\example{Se moltiplichiamo per due il valore appena trovato
($0.3413 \cdot 2 = 0.6826$) riotteniamo il ben noto risultato che la
probabilit\`a che una variabile gaussiana disti meno di una deviazione
standard dal suo valor medio \`e circa il 68\%.}

\end{exemplify}

Nel capitolo \ref{chap:Distribuzioni} abbiamo gi\`a anticipato che questo
integrale (che rappresenta la probabilit\`a di trovare una variabile normale
in forma standard entro l'intervallo $\cinterval{0}{x}$) non ha espressione
analitica e deve essere valutato numericamente.
La funzione $\erf$ si trova tabulata qui di seguito per un insieme
significativo di valori di $x$.
Il procedimento base per estrarre dalle tavole i valori di probabilit\`a
\`e stato gi\`a descritto nel capitolo \ref{chap:Distribuzioni} e \`e
sostanzialmente fondato sulla relazione, valida per una variabile Gaussiana
standard%
\footnote{
Alcuni dei programmi di analisi dati comunemente usati (ad esempio scilab e
\gnuplot) utilizzano la definizione (\ref{eq:Erf}).
In questo caso si ha, ovviamente:
$$
\prob{a \leq x \leq b} =
\frac{1}{2} \left[ \Erf\left(\frac{b}{\sqrt{2}}\right) -
\Erf\left(\frac{a}{\sqrt{2}}\right) \right]
$$
}%
:
\begin{equation}\label{eq:ProbErf}
\prob{a \leq x \leq b} = \erf(b) - \erf(a)
\end{equation}
Per una distribuzione Gaussiana con media e deviazione standard arbitrarie
la (\ref{eq:ProbErf}) pu\`o ancora essere utilizzata passando alla variabile
in forma standard (cfr. capitolo \ref{chap:Distribuzioni}).
