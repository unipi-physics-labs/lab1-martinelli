\begin{exercises}

\exercise{Ricavare le formule di propagazione dell'errore massimo nelle quattro
operazioni elementari (somma, differenza, prodotto, quoziente) a partire dalla
formula generale (\ref{eq:ErroreMassimoGenerale}).}

\exercise{Calcolare l'errore massimo sul volume del cilindro,
ricavato nell'esempio \ref{esem:VolumeCilindroRisolto} a partire dalla
formula di propagazione dell'errore massimo sul prodotto
(\ref{eq:ErroreProdotto}).}

\exercise{Applicare la formula di propagazione dell'errore massimo sul
prodotto (\ref{eq:ErroreProdotto}) all'esempio
\ref{esem:ArrotondamentoProdotto1} e discutere il risultato.

\hint{si consideri, ad esempio, $a = 1.25$, $\Delta a = 0.005$, $b = 9.4$,
$\Delta b = 0.005$.}}

\exercise{Applicare la formula di propagazione dell'errore massimo sul
prodotto (\ref{eq:ErroreProdotto}) all'esempio
\ref{esem:ArrotondamentoProdotto2}.

\hint{si consideri, ad esempio, $\Delta a = \Delta b = 5 \times 10^{-4}$.}}

\exercise{Si considerino le due grandezze:
\begin{eqnarray*}
a & = & 10.3484\\
b & = & 6.24
\end{eqnarray*}
Con quante cifre significative deve essere preso il quoziente?
Verificare che si ottiene lo stesso risultato dalla propagazione dell'errore
massimo (\ref{eq:ErroreQuoziente}).
}

\exercise{Nel prodotto tra le due grandezze:
\begin{eqnarray*}
a & = & 1.25432\\
b & = & 9.3
\end{eqnarray*}
a quante cifre significative conviene arrotondare $a$?

\hint{eseguire i prodotti arrotondando successivamente a $5$, $4$, $3$,
$2$ \ldots~cifre significative.}}

\exercise{Una certa attenzione va fatta quando i prodotti sono pi\`u di due.
Consideriamo ad esempio il caso:
$$
p = 1.234 \times 56.789 \times 11.9955 \times 5.1248
$$
Mostrare che potrebbe essere dannoso arrotondare prima della moltiplicazione
ciascun fattore a quattro cifre. Conviene partire dai fattori con pi\`u cifre
significative o da quelli con meno?}

\exercise{A quante cifre significative va arrotondato
$\pi=3.14159265$ affinch\'e le potenze $\pi^2$, $\pi^3$, $\pi^4$, $\pi^5$,
$\pi^6$ siano corrette al livello dell'1\%? E allo 0.1\%?}

\exercise{Consideriamo la grandezza:
$$
G = \frac{\sqrt{2}+\pi^2}{\sqrt{10}-\pi}
$$
Con quante cifre significative dobbiamo prendere $\sqrt{2}$,
$\sqrt{10}$ e $\pi$ se vogliamo che l'errore relativo su $G$ sia $1\%$?}

\exercise{Ripetere l'esercizio precedente nel caso in cui l'errore relativo
richiesto sia lo $0.1\%$}

\end{exercises}
