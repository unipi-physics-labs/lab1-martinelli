\chapter{Breve glossario di \gnuplot}

Questa appendice include alcuni dei comandi di \gnuplot\ pi\`u comunemente
usati, in combinazione con alcune tra le opzioni pi\`u utili.
Pu\`o servire da riferimento veloce durante l'elaborazione di dati.
Si rimanda comunque al capitolo \ref{chap:gnuplot} per informazioni pi\`u
dettagliate sul significato dei singoli comandi.

\begin{verbatim}
gnuplot> set output
gnuplot> set terminal x11
\end{verbatim}
\gnuplotcmd{Questi due comandi preparano \gnuplot\ ad operare nella modalit\`a
comunemente impiegata, in cui i grafici vengono rediretti sullo schermo.
\`E una buona abitudine includerli all'inizio ed alla fine di ogni macro.
Ricordiamo anche la variante per \windows\ \cchar{set terminal windows}.}

\begin{verbatim}
gnuplot> set terminal postscript
gnuplot> set output 'prova.eps'
\end{verbatim}
\gnuplotcmd{Questi comandi redirigono l'uscita su di un file postscript
denominato \cchar{prova.eps}. \`E essenziale ricordarsi di tornare allo
schermo con i due comandi precedenti, una volta scritto il file.}

\begin{verbatim}
gnuplot> help
\end{verbatim}
\gnuplotcmd{Visualizza le pagine di documentazione interattiva. Pu\`o
accettare come argomento comandi o parole chiave di \gnuplot\ su cui si
desidera avere informazioni.}

\begin{verbatim}
gnuplot> set title 'Titolo del grafico'
\end{verbatim}
\gnuplotcmd{Imposta il titolo del grafico, che \`e accettato come parametro
tra virgolette o tra apici.}

\begin{verbatim}
gnuplot> set xlabel 'x (cm)'
\end{verbatim}
\gnuplotcmd{Imposta l'etichetta di testo sull'asse delle $x$, passata come
parametro tra virgolette o apici.}

\begin{verbatim}
gnuplot> set ylabel 'y (cm)'
\end{verbatim}
\gnuplotcmd{Imposta l'etichetta di testo sull'asse delle $y$, passata come
parametro tra virgolette o apici.}

\begin{verbatim}
gnuplot> set xrange [0.5:1.5]
\end{verbatim}
\gnuplotcmd{Imposta gli estremi del grafico sull'asse delle $x$. Essi vengono
passati tra parentesi quadre, separati da due punti.}

\begin{verbatim}
gnuplot> set yrange [3:4]
\end{verbatim}
\gnuplotcmd{Imposta gli estremi del grafico sull'asse delle $y$. Funziona
esattamente come il comando precedente.}

\begin{verbatim}
gnuplot> set autoscale x
\end{verbatim}
\gnuplotcmd{Abilita l'impostazione automatica degli estremi dell'asse $x$
(che, in questo modo, si adattano ai dati).}

\begin{verbatim}
gnuplot> set autoscale y
\end{verbatim}
\gnuplotcmd{Abilita l'impostazione automatica degli estremi dell'asse $y$
(che, in questo modo, si adattano ai dati).}

\begin{verbatim}
gnuplot> unset key
\end{verbatim}
\gnuplotcmd{Elimina la legenda (se presente).}

\begin{verbatim}
gnuplot> set key
\end{verbatim}
\gnuplotcmd{Ripristina la legenda (se era stata eliminata). \`E il contrario
del comando precedente.}

\begin{verbatim}
gnuplot> set logscale x
\end{verbatim}
\gnuplotcmd{Seleziona la scala logaritmica sull'asse delle $x$.}

\begin{verbatim}
gnuplot> unset logscale x
\end{verbatim}
\gnuplotcmd{Ripristina la scala lineare sull'asse delle $x$ (se era
selezionata quella logaritmica). \`E il contrario del comando precedente.}

\begin{verbatim}
gnuplot> set logscale y
\end{verbatim}
\gnuplotcmd{Seleziona la scala logaritmica sull'asse delle $y$.}

\begin{verbatim}
gnuplot> unset logscale y
\end{verbatim}
\gnuplotcmd{Ripristina la scala lineare sull'asse delle $y$ (se era
selezionata quella logaritmica). \`E il contrario del comando precedente.}

\begin{verbatim}
gnuplot> lambda = 3.0
\end{verbatim}
\gnuplotcmd{Crea una variabile chiamata \cchar{lambda} e le assegna il valore
di $3.0$. Le variabili possono essere identificate con una qualsiasi
combinazione di lettere e numeri (purch\'e cominci con una lettera).
Questa sintassi \`e spesso usata quando si eseguono dei fit.}

\begin{verbatim}
gnuplot> print lambda
\end{verbatim}
\gnuplotcmd{Stampa sullo schermo il valore della variabile \cchar{lambda}
(che deve essere stata precedentemente definita).}

\begin{verbatim}
gnuplot> plot 'dati.txt' using 1:2
\end{verbatim}
\gnuplotcmd{Crea un grafico a partire dai dati contenuti nel file
\cchar{dati.txt}. In questo caso particolare la prima colonna \`e messa
sull'asse delle $x$, la seconda su quello delle $y$.}

\begin{verbatim}
gnuplot> plot 'dati.txt' using 1:2:3 with yerrorbars
\end{verbatim}
\gnuplotcmd{Come il comando precedente, ma la terza colonna \`e questa volta
interpretata coma colonna degli errori sull'asse delle $y$, che sono
correttamente riportati sul grafico.}

\begin{verbatim}
gnuplot> plot 'dati.txt' using 1:2:3:4 with xyerrorbars
\end{verbatim}
\gnuplotcmd{Come i due comandi precedenti, con le barre d'errore su entrambi
gli assi.}

\begin{verbatim}
gnuplot> plot 'dati.txt' using 1:2 with histeps
\end{verbatim}
\gnuplotcmd{I dati contenuti nel file \cchar{dati.txt} vengono rappresentati
sotto forma di istogramma.}

\begin{verbatim}
gnuplot> plot 'dati.txt' using 1:(($2)**2)
\end{verbatim}
\gnuplotcmd{Con questa sintassi sull'asse delle $y$ vengono rappresentati i
valori contenuta nella seconda colonna, elevati al quadrato.}

\begin{verbatim}
gnuplot> replot
\end{verbatim}
\gnuplotcmd{Esegue di nuovo l'ultimo comando \cchar{plot} che \`e stato
impartito nella \emph{shell} di \gnuplot.}

\begin{verbatim}
gnuplot> f(x) = a*sin(x)
\end{verbatim}
\gnuplotcmd{Definisce una funzione della variabile indipendente $x$ (che non
necessita di essere definita a sua volta). La funzione pu\`o poi essere usata
in un fit.}

\begin{verbatim}
gnuplot> fit f(x) 'dati.txt' using 1:2 via a
\end{verbatim}
\gnuplotcmd{Esegue un fit ai dati contenuti nel file \cchar{dati.txt} con la
funzione (precedentemente definita) $f(x)$. Il parametro $a$ \`e lasciato
libero di variare (ricordiamo che la sua inizializzazione \`e essenziale per
una corretta convergenza del fit).}

\begin{verbatim}
gnuplot> fit f(x) 'dati.txt' using 1:2:3 via a
\end{verbatim}
\gnuplotcmd{Come il comando precedente, ma in questo caso i punti $(x_i, y_i)$
sono pesati con gli errori $\Delta y_i$, contenuti nella colonna numero $3$
del file di dati.}

%\begin{verbatim}
%gnuplot>
%\end{verbatim}
%\begin{quotation}
