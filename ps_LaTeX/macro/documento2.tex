\documentclass[11pt, a4paper]{article}
\title{Il mio primo documento}
\author{Luca Baldini}
\date{20 Giugno 2006}

\begin{document}
\maketitle

\section{Introduzione}
Questo vuole essere un esempio di documento realistico, con lo scopo
di mostrare alcune tra le funzionali\`a di base di \LaTeX.

\section{Un nuovo capitolo}
Una prima cosa da notare \`e che, una volta dichiarate le sezioni del
documento, \LaTeX\ si occupa atomaticamente della numerazione. Questo
pu\`o risultare estrememente utile nel caso in cui si voglia aggiungere
una nuova sezione in mezzo ad un documento in stato avanzato di stesura
(non \`e necessario rinumerare ci\`o che viene dopo, \LaTeX\ fa
tutto da solo). 

\subsection{Una sotto-sezione}
Come vedete esiste anche un livello pi\`u \emph{basso} di sezionamento
del documento.

\subsection{Ancora una sottosezione}
Anche a questo livello la numerazione \`e automatica. Questo consente
anche, in documenti pi\`u complessi, la generazione automatica dell'indice.

\section{Conclusioni}
Per il momento ci fermiamo qui, abbiamo abbastanza di cui discutere\ldots

\end{document}
