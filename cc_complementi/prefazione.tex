\thispagestyle{empty}

\chapter*{Prefazione}
\pdfbookmark[0]{Prefazione}{prefazione}

Il materiale raccolto in queste dispense proviene sostanzialmente
da due nuclei fondamentali: le dispense originali tratte dalle lezioni
del corso di Esperimentazioni di Fisica I per Fisici,
tenute da L.~M. presso l'Universit\`a degli Studi di Pisa,
ed una breve introduzione all'uso del calcolatore nei laboratori didattici,
originariamente distribuita direttamente tra gli studenti del corso.
Bench\'e molte altre cose si siano via via aggiunte con il passare del tempo,
l'organizzazione concettuale del materiale riflette la presenza di questi due
nuclei originali.

Nella prima parte si introducono alcuni tra i fatti fondamentali relativi
alla propagazione delle incertezze nelle misure fisiche, al calcolo delle
probabilit\`a (con particolare attenzione alle distribuzioni pi\`u spesso
usate in fisica) ed al trattamento statistico dei dati (incluse le pi\`u
comuni tecniche di \emph{fit}).
Questa parte contiene inoltre una trattazione, sia pure sommaria, di alcuni
argomenti appena pi\`u avanzati (correlazione, distribuzione t, cambiamento di
variabile nelle funzioni di distribuzione) che, pur non essendo cos\`i
pesantemente usati nelle esperienze del primo anno, devono a buon diritto far
parte del bagaglio culturale del fisico medio.
Notiamo, per inciso, che, come i numerosi esempi dimostrano, la discussione \`e
tendenzialmente focalizzata sul significato fisico delle nozioni introdotte,
pi\`u che sul formalismo, e che, come \`e ovvio, il tono dell'esposizione non
\`e sempre completamente rigoroso.

Lo scopo fondamentale della seconda parte \`e quello di rendere il meno
traumatico possibile l'incontro con il calcolatore degli studenti del primo
anno e, anche l\`a dove questo incontro si rivela tutto sommato indolore,
promuovere un uso \emph{consapevole} del calcolatore stesso (cosa spesso
tutt'altro che scontata).
Questo include (ma non \`e limitato a) il conseguimento della capacit\`a di
utilizzare il \emph{computer} per rappresentare ed analizzare dati
sperimentali e (perch\'e no?) presentarli in una forma esteticamente
appetibile.

\caution{Il simbolo di \emph{curva pericolosa} che ricorre occasionalmente,
nel seguito, ha lo scopo di attrarre l'attenzione su alcuni passaggi
particolarmente delicati che meritano di essere letti almeno due volte.}

Gli autori desiderano ringraziare sinceramente Alessandro Strumia,
che per primo ha trascritto in \LaTeX\ il contenuto originale delle lezioni di
Esperimentazioni di Fisica I e tutti i docenti (in particolare
Stefano Sanguinetti) e gli studenti (in particolare Francesca Menozzi) che,
segnalando errori ed imprecisioni, hanno contribuito a rendere queste dispense
pi\`u comprensibili ed accurate.

\begin{flushright}
\vspace{0.5cm}
Pisa, 13 Agosto 2007\\
\vspace{0.3cm}
L. Martinelli\\
L. Baldini
\end{flushright}
